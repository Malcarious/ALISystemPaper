\documentclass[amtd, online, hvmath]{copernicus}

\begin{document}\hack{\sloppy}

\title{The Aerosol Limb Imager: acousto-optic imaging of limb scattered
sunlight for stratospheric aerosol profiling}

\Author{B.~J.}{Elash}
\Author{A.~E.}{Bourassa}
\Author{P.~R.}{Loewen}
\Author{N.~D.}{Lloyd}
\Author{D.~A.}{Degenstein}

\affil{Institute of Space and Atmospheric Studies, Saskatchewan,
Canada}


\correspondence{B.~J.~Elash (brenden.elash@usask.ca)}

\runningtitle{The Aerosol Limb Imager}
\runningauthor{B.~J.~Elash et~al.}


\received{26~October~2015}
\accepted{30~November~2015}
\published{}


\firstpage{1}

\maketitle


\begin{abstract}
  The Aerosol Limb Imager (ALI) is an optical remote sensing
  instrument designed to image scattered sunlight from the atmospheric
  limb. These measurements are used to retrieve spatially resolved
  information of the stratospheric aerosol distribution, including
  spectral extinction coefficient and particle size. Here we present
  the design, development and test results of an ALI prototype
  instrument. The long term goal of this work is the eventual
  realization of ALI on a~satellite platform in low earth orbit, where
  it can provide high spatial resolution observations, both in the
  vertical and cross-track. The instrument design uses a~large
  aperture Acousto-Optic Tunable Filter (AOTF) to image the sunlit
  stratospheric limb in a~selectable narrow wavelength band ranging
  from the visible to the near infrared. The ALI prototype was tested
  on a~stratospheric balloon flight from the Canadian Space Agency
  (CSA) launch facility in Timmins, Canada, in
  September~2014. Preliminary analysis of the hyperspectral images
  indicates that the radiance measurements are of high quality, and we
  have used these to retrieve vertical profiles of stratospheric
  aerosol extinction coefficient from 650--1000\,\unit{nm}, along with
  one moment of the particle size distribution. Those preliminary
  results are promising and development of a~satellite prototype of
  ALI within the Canadian Space Agency is ongoing.
\end{abstract}

\end{document}

\endinput

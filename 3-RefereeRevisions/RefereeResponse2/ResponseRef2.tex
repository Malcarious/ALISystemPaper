\documentclass[12pt, notitlepage]{article}

\usepackage[letterpaper, margin=1in]{geometry}
\usepackage{graphicx}

\setlength\parindent{0pt}

\title{Interactive comment to ``The Aerosol Limb Imager: acousto-optic imaging of limb scattered
sunlight for stratospheric aerosol profiling'' by B. J. Elash et al}

\author{B.~J. Elash et al. (brenden.elash@usask.ca)}

\begin{document}

\begin{titlepage}
\maketitle
\end{titlepage}


We would like to thank the referee for their helpful comments and suggestions. Below are the referee's comments in italics followed by our reply.

\hrulefill

\textit{Sect. 1, 2nd paragraph:}

\textit{``the observations are essentially always limited to some degree'' - This statement is
unclear, but maybe it means that no single measurement technique provides the full
range of aerosol properties unambiguously?}\\

\textbf{Reply:} The text has been modified to improve clarity, and the sentence now reads ``\ldots although due to the variability of physical composition
and particle size, no single measurement technique can determine the full range of aerosol properties unambiguously.''.

\hrulefill

\textit{Sect. 1, 2nd paragraph:}

\textit{``. . . Charlson et al., 1969); acquire . . .'' – the semicolon should be removed.}\\

\textbf{Reply:} Corrected.

\hrulefill

\textit{Sect. 1, 3rd paragraph:}

\textit{``. . . there are challenges associated with comparing the retrieved extinction profiles
to other microphysical parameters'' – Do you mean that it is difficult to derive other
microphysical parameters from the retrieved extinction profiles?}\\

\textbf{Reply:} It is difficult to determine the microphysical parameters which can make comparisons between instruments  difficult. The text has been modified with the following to make this more clear: ``\ldots have generally compared well with ground based and
in-situ measurements, although there are challenges associated with
determining microphysical
parameters and comparison between instruments can be challenging.''

\hrulefill

\textit{Sect. 1, 3rd paragraph:}

\textit{``. . . allowing for straight forward retrieval'' – should be one word (straightforward).}\\

\textbf{Reply:} Corrected.

\hrulefill

\textit{Sect. 1, 4th paragraph:}

\textit{It appears that the SAGE – OSIRIS merged aerosol dataset is described as an ``essentially
continuous long term record'' earlier. In that case, it might be useful to quantify
how consistent retrievals from the 2 missions are, rather than simply reporting that they
agree ``relatively well.''}\\

\textbf{Reply:} It has been noted and the relatively well has been quantified to generally within 30\% below 30\,km within the text (Bourassa et al., 2012; Rieger et al., 2015).

\hrulefill

\textit{Sect. 1, 6th paragraph:}

\textit{``. . . has been studied extensively and somewhat controversially. . .'' – I agree that controversy
has arisen, but don't like this wording. The conclusions are controversial, but
the issue hasn't really been ``studied controversially,'' has it?}\\

\textbf{Reply:} The sentence has been reworded for clarity to the following ``the transport and origin of which has been
studied extensively and the conclusions are somewhat controversial''.

\hrulefill

\textit{Sect. 1, 7th paragraph:}

\textit{It might be useful to measure the CATS mission, which recently began operation of a
lidar on the International Space Station.}\\

\textbf{Reply:} A reference to the CATS mission has been added at the end of this paragraph with the following sentence. ``The lidar instrument Cloud Aerosol Transport System (CATS) (Chuang et al., 2013) has been placed on the international space station in 2015 for an expected mission lifetime of three years.''

\hrulefill

\textit{Sect. 2.1, 2nd paragraph:}

\textit{All equations are labeled oddly: The equation number appears to be part of the equation
text (rather than right-justified). This is difficult to read and should be fixed.}\\

\textbf{Reply:} This has been fixed.

\hrulefill

\textit{Sect. 2.1, 3rd paragraph:}

\textit{``. . . acousto-optic diffraction angle is not constant angle with wavelength. . .'' – Do you
mean ``. . . angle varies with wavelength. . .''?}\\

\textbf{Reply:} Yes that is the meaning of the sentence and the text has been altered for clarity. It now reads ``The acousto-optic diffraction angle varies as the filtered wavelength is changed\ldots''.

\hrulefill

\textit{Sect. 2.2, 1st paragraph:}

\textit{``We also attempted to pay careful attention to stray light. . .'' Unless you have something
you'd like to confess in this paragraph, I think you can honestly say that ``We paid careful
attention to stray light. . .''}\\

\textbf{Reply:} The change has been made in the text.

\hrulefill

\textit{Sect. 2.2, 4th paragraph:}

\textit{``. . . this wavelength dependant change is negligible. . .'' – should say ``dependent'', and
could you quantify how small the change is?}\\

\textbf{Reply:} The correction has been made. And this change is typically less than a micrometer for the entire wavelength range of ALI and the following has been added to the text, ``\ldots this wavelength dependent change is less than a micrometer for the current ALI design and is considered negligible.''.

\hrulefill

\textit{Sect. 2.2, 6th paragraph:}

\textit{``. . ., a not insignificant fraction of light'' – again, could you quantify this?}\\

\textbf{Reply:} This is quantified in the following sentence. ``The diffracted extraordinary signal comprises at
most a~$\sim 10$\,nm bandpass fraction of one polarization such
that the unabsorbed broadband signal from the polarizers can be on the
same order of intensity as the diffracted signal.''

\hrulefill

\textit{Sect. 2.2, 6th paragraph:}

\textit{``. . . extraordinary signal compresses at most. . .'' – do you mean ``comprises''?}\\

\textbf{Reply:} Corrected.

\hrulefill

\textit{Sect. 4.2, 1st paragraph:}

\textit{This paragraph reports that an unexpectedly large amount of stray light was seen at
high altitudes in the field measurements, but neither the magnitude of the expected
stray light level nor the observed excess stray light level is clearly quantified here. I
would like to see both numbers estimated.}\\

\textbf{Reply:} The text has been altered to include this information and has added the following sentence: ``For the high altitudes in the range of 27 to 30\,km the expected ratio of signal to stray light was estimated to be between 2-3 but for the campaign the ratio of signal to stray light for some regions dropped down slightly below one.''

\hrulefill

\textit{Sect. 4.4, 2nd paragraph:}

\textit{``There are also several possible systematic errors not accounted for in the inversion
including the choice of retrieval altitude ranges, particle size composition and distributions,
stray light, and the high altitude aerosol load.''}\\

\textit{This sentence is confusing in a few respects:}\\

\textit{1. Couldn't you adjust several of these parameters (such as retrieval altitude range,
particle size distribution, and particle composition) to use the same assumption in the
ALI retrieval as in the OSIRIS retrieval? Was this tried, in an attempt to sort out which
differences might be most significant?}\\

\textbf{Reply:} This was attempted but was not possible. The same particle size distribution and composition
that is used for the OSIRIS retrieval was also used in the ALI retrieval.  However, two factors limited the ability to use the same assumptions as OSIRIS. First, OSIRIS uses a normalization altitude that is above the usable range for ALI resulting in ALI always having a lower upper bound than OSIRIS. Second, OSIRIS uses a wavelength normalization at 470\,nm when calculating its measurement vector (Bourassa et al., 2012). However ALI does not extend down to this wavelength range and as such a short wavelength normalization could not be accounted for in the analysis. Between these two effects the same assumptions for OSIRIS could not be replicated for ALI and the retrievals would have some biases that can not be reconciled.\\

\textit{2. How is the high altitude aerosol load a source of systematic differences? Again,
couldn't the same assumed value be used for both retrievals? (I assume we're talking
about the aerosol above the retrieval range.)}\\

\textbf{Reply:} Yes we are referring to the aerosol above the retrieval range. The same a priori that is used for OSIRIS is used for ALI. However, just setting the aerosol above and below ALI's retrieval range to the OSIRIS value causes discontinuities in the aerosol profile that leads to non-physical or unconverged profiles for the retrievals. Instead, the retrieval scales the aerosol profile above and below the retrieval range by the nearest scaling factor with the valid retrieval range determined by MART. Altering the slope of the a priori profile above the range was attempted to try to determine the effect and a maximum change of 5\% was noted in the retrieved aerosol profile.\\

\textit{3. The phrase ``particle size composition and distributions'' is confusing. Maybe some
commas are missing, but I think you mean particle size distribution and particle composition.
It might be useful to state the scattering angle for the solar beam for the ALI observations,
and to compare its value to the same value for the OSIRIS observations.}\\

\textbf{Reply:} The phase ``particle size composition and distributions'' has been replace with ``particle size distribution and particle composition'' to help clarify this statement. The scattering angle for ALI and OSIRIS have been added to the text as well as a short comparison with the addition of the following sentences. ``The solar scattering angle for a measurement can also have an effect on the retrieved profile due to sensitivity to the scattering cross sections from the particle size distributions. For the ALI image the solar scattering angle is 98\,degrees and for the five OSIRIS scans they are 77, 89, 90, 91, 92, 93\,degrees. With the exception of the forward scatter angles of 77 and 89\,degrees from OSIRIS, the scattering angles between OSIRIS and ALI are similar and should not cause a large effect on the retrieved profiles.''\\

\hrulefill

\textit{Sect. 4.4, 3rd paragraph:}

\textit{It sounds like the same aerosol size distribution is assumed at all altitudes in the retrieval,
so can you explain what it means when an Angstrom coefficient that varies with
height is observed in the retrieved extinction profiles? And how does the observed
value compare to the value that Mie theory would predict for the assumed size distribution?}\\

\textbf{Reply:} In the retrieval the Angstr\"{o}m coefficient for each altitude is calculated independently and allows for each altitude to have a unique value. However when the state of the model is updated, the median of the altitude dependent Angstr\"{o}m profile is used as the value of the particle size distribution for the next iteration. This results in retrievals that yield altitude dependent coefficient but use the median in the next iteration of the retrieval. The method used here provides stability and a full discussion of particle size retrievals is out of scope for the work presented here. For details on particle size algorithms see Rault and Loughman (2013) and  Rieger et al. (2014). The observed Angstr\"{o}m coefficient is approximately 2.7 whereas the assumed distribution is 2.3. The fine mode average of Angstr\"{o}m coefficients from work derived from Deshler et al. (2003) are 2.1 to 3.4 and ALI's determined value is within the expected range as stated in the text. The addition of the retrieved median Anstr\"{o}m coefficient and the Deshler et al. (2003) range has been added to the text.

\hrulefill

\textit{Sect. 5, 2nd paragraph:}

\textit{``. . .should be tacked in future iterations. . .'' – should this say ``tackled''?}\\

\textbf{Reply:} Corrected.

\hrulefill

Bourassa,~A.~E., Rieger,~L.~A., Lloyd,~N.~D., and Degenstein,~D.~A.:
Odin-OSIRIS stratospheric aerosol data product and SAGE III intercomparison,
Atmos. Chem. Phys., 12, 605--614,
doi:10.5194/acp-12-605-2012,
2012.\\

Rault,~D.~F. and Loughman,~R.~P.: The OMPS limb profiler environmental data
record algorithm theoretical basis document and expected performance, IEEE T.
Geosci. Remote, 51, 2505--2527, 2013.\\

Rieger,~L.~A., Bourassa,~A.~E., and Degenstein,~D.~A.: Stratospheric aerosol
particle size information in Odin-OSIRIS limb scatter spectra, Atmos. Meas.
Tech., 7, 507--522,
doi:10.5194/amt-7-507-2014,
2014.\\

Rieger,~L.~A., Bourassa,~A.~E., and Degenstein,~D.~A.: Merging the OSIRIS and
SAGE II stratospheric aerosol records,~J. Geophys. Res., 120, 8890--8904,
doi:10.1002/2015JD023133,
2015.\\

\end{document}


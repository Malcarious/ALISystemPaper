\documentclass[12pt, notitlepage]{article}

\title{Interactive comment to ``The Aerosol Limb Imager: acousto-optic imaging of limb scattered
sunlight for stratospheric aerosol profiling'' by B. J. Elash et al}

\author{B.~J. Elash et al. (brenden.elash@usask.ca)}

\begin{document}

\begin{titlepage}
\maketitle
\end{titlepage}


We would like to thank the referee for their helpful comments and suggestions. Below are the referee's comments in italics followed by our reply.

\hrulefill

\textit{Sect. 1, 2nd paragraph:}

\textit{``the observations are essentially always limited to some degree'' - This statement is
unclear, but maybe it means that no single measurement technique provides the full
range of aerosol properties unambiguously?}

\textbf{Reply:} The text has been modified to improve clarity, and the sentence now reads ``...although due to the variability of physical composition
and particle size, no single measurement technique can fully determine the full range of aerosol properties unambiguously.''.

\hrulefill

\textit{Sect. 1, 2nd paragraph:}

\textit{``. . . Charlson et al., 1969); acquire . . .'' – the semicolon should be removed.}

\textbf{Reply:} It has already been removed in the AMTD published paper.

\hrulefill

\textit{Sect. 1, 3rd paragraph:}

\textit{``. . . there are challenges associated with comparing the retrieved extinction profiles
to other microphysical parameters'' – Do you mean that it is difficult to derive other
microphysical parameters from the retrieved extinction profiles?}

\textbf{Reply:} Partially, it is difficult to determine the microphysical parameters as well comparisons can be vastly different between instruments. The text has been modified to make this clearer.

\hrulefill

\textit{Sect. 1, 3rd paragraph:}

\textit{``. . . allowing for straight forward retrieval'' – should be one word (straightforward).}

\textbf{Reply:} Corrected.

\hrulefill

\textit{Sect. 1, 4th paragraph:}

\textit{It appears that the SAGE – OSIRIS merged aerosol dataset is described as an ``essentially
continuous long term record'' earlier. In that case, it might be useful to quantify
how consistent retrievals from the 2 missions are, rather than simply reporting that they
agree ``relatively well.''}

\textbf{Reply:} It has been noted and the relatively well has been quantified to generally within 15\% within the text.

\hrulefill

\textit{Sect. 1, 6th paragraph:}

\textit{``. . . has been studied extensively and somewhat controversially. . .'' – I agree that controversy
has arisen, but don't like this wording. The conclusions are controversial, but
the issue hasn't really been ``studied controversially,'' has it?}

\textbf{Reply:} The sentence has been reworded for clarity to the following ``the transport and origin of which has been
studied extensively and the conclusions are somewhat controversially''.

\hrulefill

\textit{Sect. 1, 7th paragraph:}

\textit{It might be useful to measure the CATS mission, which recently began operation of a
lidar on the International Space Station.}

\textbf{Reply:} The CATS mission was unknown to us and and been added at the end of this paragraph has a current instrument that measures aerosol.

\hrulefill

\textit{Sect. 2.1, 2nd paragraph:}

\textit{All equations are labeled oddly: The equation number appears to be part of the equation
text (rather than right-justified). This is difficult to read and should be fixed.}

\textbf{Reply:} This has been fixed in the AMTD published paper.

\hrulefill

\textit{Sect. 2.1, 3rd paragraph:}

\textit{``. . . acousto-optic diffraction angle is not constant angle with wavelength. . .'' – Do you
mean ``. . . angle varies with wavelength. . .''?}

\textbf{Reply:} Yes that is the meaning of the sentence and the text has been altered for clarity. It now read ``The acousto-optic diffraction angle varies as the filtered wavelength is changed...''.

\hrulefill

\textit{Sect. 2.2, 1st paragraph:}

\textit{``We also attempted to pay careful attention to stray light. . .'' Unless you have something
you'd like to confess in this paragraph, I think you can honestly say that ``We paid careful
attention to stray light. . .''}

\textbf{Reply:} The changed has been made in the text.

\hrulefill

\textit{Sect. 2.2, 4th paragraph:}

\textit{``. . . this wavelength dependant change is negligible. . .'' – should say ``dependent'', and
could you quantify how small the change is?}

\textbf{Reply:} The correction has been made. And this change is typically less than a micron for the entire wavelength range of ALI.

\hrulefill

\textit{Sect. 2.2, 6th paragraph:}

\textit{``. . ., a not insignificant fraction of light'' – again, could you quantify this?}

\textbf{Reply:}This is quantified in the following sentence. ``The diffracted extraordinary signal comprises at
most a~$\sim 10$\,nm bandpass fraction of one polarization such
that the unabsorbed broadband signal from the polarizers can be on the
same order of intensity as the diffracted signal.''

\hrulefill

\textit{Sect. 2.2, 6th paragraph:}

\textit{``. . . extraordinary signal compresses at most. . .'' – do you mean ``comprises''?}

\textbf{Reply:} Corrected.

\hrulefill

\textit{Sect. 4.2, 1st paragraph:}

\textit{This paragraph reports that an unexpectedly large amount of stray light was seen at
high altitudes in the field measurements, but neither the magnitude of the expected
stray light level nor the observed excess stray light level is clearly quantified here. I
would like to see both numbers estimated.}

\textbf{Reply:}

\hrulefill

\textit{Sect. 4.4, 2nd paragraph:}

\textit{``There are also several possible systematic errors not accounted for in the inversion
including the choice of retrieval altitude ranges, particle size composition and distributions,
stray light, and the high altitude aerosol load.''}

\textit{This sentence is confusing in a few respects:}

\textit{1. Couldn't you adjust several of these parameters (such as retrieval altitude range,
particle size distribution, and particle composition) to use the same assumption in the
ALI retrieval as in the OSIRIS retrieval? Was this tried, in an attempt to sort out which
differences might be most significant?}

\textbf{Reply:} This was not entirely possible and was attempted. The same particle size distribution
that is used for OSIRIS retrieval was used in the ALI retrieval. And the retrieved particle size is vary
similar to the OSIRIS assumed distribution (OSIRIS assumes a mode radius of 0.08 $\mu$m whereas ALI found a size of 0.077 $\mu$m).  However, the limiting factor leading to difficulty
in using this method is that the OSIRIS retrieval uses a higher altitude for normalization which
is outside of ALI's usable range. This results in always using a lower normalization point and results in a difference
that could not be rectified by changed ALI's altitude range.

\textit{2. How is the high altitude aerosol load a source of systematic differences? Again,
couldn't the same assumed value be used for both retrievals? (I assume we're talking
about the aerosol above the retrieval range.)}

\textbf{Reply:} Yes we are referring about the aerosol above the retrieval range. The same a priori that is used for OSIRIS is used for ALI. However, just setting the aerosol above and below ALI retrieval range to the OSIRIS value causes discontinuities in the aerosol profile that cause non-physical profiles for the retrievals. Instead the retrieval scales the aerosol profile above and below the retrieval range to by the nearest scaling factor determined by MART within the valid range. Altering the slope of the a priori profile above the range was attempted to try to determine the effect and at most a 5\% change was noted in the retrieved aerosol profile.

\textit{3. The phrase ``particle size composition and distributions'' is confusing. Maybe some
commas are missing, but I think you mean particle size distribution and particle composition.
It might be useful to state the scattering angle for the solar beam for the ALI observations,
and to compare its value to the same value for the OSIRIS observations.}

\textbf{Reply:} The phase ``particle size composition and distributions'' has been replace with ``particle size distribution and particle composition'' to help clarify this statement. The scatter angle for ALI and OSIRIS have been added to the text and well as a short comparison.
\hrulefill

\textit{Sect. 4.4, 3rd paragraph:}

\textit{It sounds like the same aerosol size distribution is assumed at all altitudes in the retrieval,
so can you explain what it means when an Angstrom coefficient that varies with
height is observed in the retrieved extinction profiles? And how does the observed
value compare to the value that Mie theory would predict for the assumed size distribution?}

\textbf{Reply:} In the retrieval the Angstr\"{o}m coefficient for each altitude is calculated independently and allows for each altitude to have a unique value. However when the state of the model is update the median of the altitude dependent Angstr\:{o}m profile is used as the value of the particle size distribution for the next iteration. This results in retrievals that yield altitude dependent coefficient but uses the median in the next iteration of the retrieval. A varying particle size distributions with respect to height in the retrieval algorithm result in instability in the method thus why the median was used. The observed Angstr\"{o}m coefficient is approximately 2.7 whereas the assumed distribution is 2.3. The fine mode average of Angstr\"{o}m coefficients from work derived from Deshler et al. (2003) are 2.1 to 3.4 and ALI's determined value is within the expected range as stated in the text. The addition of the retrieved median Anstr\:{o}m coefficient and the Desher et al. (2003) range has been added to the text.

\hrulefill

\textit{Sect. 5, 2nd paragraph:}

\textit{``. . .should be tacked in future iterations. . .'' – should this say ``tackled''?}

\textbf{Reply:} Correct in the AMTD published version.

\hrulefill



\end{document}


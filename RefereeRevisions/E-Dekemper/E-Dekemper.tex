\documentclass[12pt, notitlepage]{article}

\title{Interactive comment to ``The Aerosol Limb Imager: acousto-optic imaging of limb scattered
sunlight for stratospheric aerosol profiling'' by B. J. Elash et al}

\author{B.~J. Elash et al. (brenden.elash@usask.ca)}

\begin{document}

\begin{titlepage}
\maketitle
\end{titlepage}


We would like to thank the referee for their helpful comments and suggestions. Below are the referee's comments in italics followed by our reply.

\hrulefill

\textit{1) I am not sure that the introduction section should be so long. To me, regarding the
scope of the paper, the not-so-concise discussion on previous limb missions having
measured stratospheric aerosols is not helping in appreciating the work done here.
Shortening this part could free some space for some missing information in the calibration
section or for the error analysis.}

\textbf{Reply:}

\hrulefill

\textit{2) p. 13290, l. 23-24: the width of the spectral transmission function of an AOTF
is something which is frozen at the manufacturing step, i.e. when the crystal cutting
angles are frozen. In that sense, there is no such thing like typical bandwidths, as one
can design an AOTF with a 5-10 times narrower or broader bandpass.}

\textbf{Reply:}

\hrulefill

\textit{3) p. 13291, l. 24: The acoustic wave in this kind of device is not a standing wave. In
TeO2 AOTF, it is a shear wave mostly absorbed at the opposite end of the crystal.}

\textbf{Reply:}

\hrulefill

\textit{4) AOTF design. There is an extensive discussion on the selection of the most appropriate
optical design which is well argued. However, I wonder why the rear facet of the
AOTF was cut such as depicted in fig.2 (by the way, replace "standing RF" by "acoustic"
in this figure). From the moment it is decided to work with the e-light as input, a
better configuration could have been found where the diffracted beam remains parallel
to the incident axis and a larger angular separation is achieved with the 0-order. This is
important because with a half-FOV of 3◦, and taking into account your drawing and the
fact that the diffracted beam leaves the crystal with an angle of 2.7◦, there should be a
significant overlapping of the 0th and 1st orders... Could you better justify this design
choice in the text?}

\textbf{Reply:}

\hrrulefill

\textit{5) I think the section 2.2 should contain a proper mathematical description of the radiometric
model of the instrument, including the spectral transmission function, the
polarization sensitivity and other effects such as PRNU. This would certainly help in
understanting the impact of the calibration uncertainties when discussing the error budget.}

\textbf{Reply:} 

\hrulefill

\textit{6) p. 13299, l. 14-15: I would not say that 1.2nm is much less than the AOTF spectral
resolution. You indeed performed a characterization of the spectral transmission
function of the AOTF with a not-exactly monochromatic light source. In the end you got
a result (fig.6a) which is the convolution of the incident light spectrum and the AOTF
response. The typical sidelobes are not completely resolved, but this is not really an
issue for your calibration as the results seem perfectly in line with standard AOTF performance.
I would recommend next time to work with sharp emission lines or laser lines
at some selected wavelengths, and rely on the physics of acousto-optic interaction to
extrapolate the AOTF response function between the calibration points.}

\textbf{Reply:} 

\hrulefill

\textit{7) Section 3.1: Why didn’t you use a physical model to fit the experimental data with
the AOTF tuning curve? This would provide a better understanding of the overall instrument.
Also, as the F(lambda) relationship is dependent on the crystal temperature,
it would be usefull to compare the temperature in the lab when the calibration of the
instrument was done with the temperature of the crystal during the flight. Again, a
physical model of the AOTF would help in extrapolating the calibration to other working
temperatures. The reported 0.1\% error in the fit can yield an uncertainty as high as
1nm. A 10◦C shift of temperature would also yield a 1nm drift. Is this still tolerable for
your measurements? More details on the precision of the wavelength selection would
be appreciated.}

\textbf{Reply:} 

\hrulefill

\textit{8) Section 3.2: It is mentioned that a diffraction efficiency of 54-64\% is observed, but
nothing is said concerning the power applied to the transducer of the crystal. Also,
these values appear quite low compared to typical DE for TeO2 easily reaching 90\%.
Moreover the method described neglects the attenuation by the crystal itself, and it is
not clear if the incident light was initially polarized. I would recommend to re-write this
section such that one can better understand how these values could be obtained.}

\textbf{Reply:} 

\hrulefill

\textit{9) Stray light: Cycling between the ON/OFF state of the AOTF is probably a unique
feature of the AOTF which is well emphasized in the text. However, knowing how
complex straylight characterization can be, I wonder why all these efforts were made
as in the end, the problem is mostly solved by the ON/OFF approach. The only effect
which is not solved by the ON/OFF method is the straylight generated after the AOTF.
Is there something that can be said on this based on the characterization that was
performed?}

\textbf{Reply:} 

\hrulefill

\textit{10) Relative flat fielding: It is not clear which setup was used to create the radiometric
flat field, and if the complete FOV was illuminated. From what can be read, I understand
that only sub-sections of the detector were illuminated, so it is not clear how the
response of pixels looking at the bottom of the scene can be related to the response of
the pixels looking above... This is important as you perform a spatial normalization in
the processing algorithm. A mathematical radiometric model would help understanding
what has been done. I would suggest to re-write this section in order to explain how
the setup looked like, with which accuracy for the flatness of the radiometric field, and
how does it impact the final product.}

\textbf{Reply:} 

\hrulefill

\textit{11) Conclusion: Taking into account the impressive amount of work that has been done
in this work, I would have expected some more discussion in the concluding section.
From what can be read in this section, further improvements of the instrument would
only consist in reaching absolute radiometric calibration, and a better flat fielding. I am
not convinced that this will significantly reduce the error bars (50\% at 1 sigma). Actually
the shortness of the conclusion reflects the absence of a detailed error budget. This is
probably my main concern with the manuscript: due to the absence of a mathematical
model of the instrument, it is not possible to understand the amplitude of the different
errors, and the results presented in fig.12 cannot really be interpreted.}

\textbf{Reply:} 

\hrulefill




\end{document}


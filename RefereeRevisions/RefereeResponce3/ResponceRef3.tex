\documentclass[12pt, notitlepage]{article}

\title{Interactive comment to ``The Aerosol Limb Imager: acousto-optic imaging of limb scattered
sunlight for stratospheric aerosol profiling'' by B. J. Elash et al}

\author{B.~J. Elash et al. (brenden.elash@usask.ca)}

\begin{document}

\begin{titlepage}
\maketitle
\end{titlepage}


We would like to thank the referee for their helpful comments and suggestions. Below are the referee's comments in italics followed by our reply.

\hrulefill

\textit{Equations should have numbers. Now some of them have random identification numbers.}

\textbf{Reply:} This has been corrected in the AMTD published version.

\hrulefill

\textit{p. 8, l. 2: Telecentric and telesopic systems. I am not familiar with these terms. Perhaps you could define them briefly.}

\textbf{Reply:} Brief descriptions of the terms were added. ``... telecentric and telesopic systems. The telecentric system uses a layout that removes perspective from the image and object plane and the telescoptic system uses a telescope as the front end optics.''

\hrulefill

\textit{Sec. 3.3: Please provide some quantitative estimates of the magnitude of the stray light compared to the signal.}

\textbf{Reply:}Using an average of the entire FOV a signal to noise ratio of 40 is noted. A sentence has been added into section 3.3.

\hrulefill

\textit{p.17, l. 16: The value of z\_ref?}

\textbf{Reply:} The following sentence has been modified to include the typical values of $z_{ref}$. ``For the ALI measurements, the highest
possible tangent altitude where the signal is above the noise threshold is
approximately 30\,km tangent height and typical values for $z_{ref}$ were between 27 and 30\,km''

\hrulefill

\textit{p.17, l. 16: Perhaps you should differentiate the observed values from the modeled values by improving notation (`m' or `model',...).}

\textbf{Reply:} The notation model has been added to the equation.

\hrulefill

\textit{p. 17, l. 28: Is MART better than, for example, Levenberg-Marquatd minimization?
What is the function you minimize by MART? Is it quadratic distance (y\_obsy\_model)**2
or something else?}

\textbf{Reply:} The MART method minimizes the function $y_{obs}/y_{mod}*\ln(y_{obs}/y_{mod})$. For application
used here MART and Levenberg-Marquatd return similar results. MART was selected since the OSIRIS aerosol
product uses MART and would help to negate errors from algorithm difference in comparing the results.

\hrulefill

\textit{Fig. 7: What are the thin horizontal and vertical lines?}

\textbf{Reply:} No thin horizontal or vertical lines are noted in the figure produced for the paper.

\hrulefill

\textit{Fig. 8: Fig. (a) looks very dark.}

\textbf{Reply:} Fig. 8 (a) brightness has been increased by 20\% and makes the image easier to read and view. See supplement for updated figure.

\hrulefill

\textit{Fig. 10. Provide the zenith angle step used to generate the dashed and solid lines.}

\textbf{Reply:} For the measurements during the mission a zenith angle step of approximately 2 degrees occurred.
Dashed lines represent solar zenith angles grater than 90 degrees, solid line are profiles with solar zenith angles less than 90.
A sentence in the figure caption has been added to include this information.

\hrulefill

\textit{p.20, l.28: Tack or tackle?}

\textbf{Reply:} Corrected in AMTD published version.

\hrulefill




\end{document}

